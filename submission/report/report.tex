\documentclass[10pt,conference,compsocconf]{IEEEtran}

\usepackage{hyperref}
\usepackage{graphicx}	% For figure environment


\begin{document}
\title{Higgs Boson Challenge : Finding evidence of the particle's presence using Machine Learning}

\author{
  Niccolo Sacchi, Antonio [???] & Valentin Nigolian\\
  \textit{Department of Computer Science, EPFL Lausanne, Switzerland}
}

\maketitle

\begin{abstract}
  A critical part of scientific discovery is the
  communication of research findings to peers or the general public.
  Mastery of the process of scientific communication improves the
  visibility and impact of research. While this guide is a necessary
  tool for learning how to write in a manner suitable for publication
  at a scientific venue, it is by no means sufficient, on its own, to
  make its reader an accomplished writer. 
  This guide should be a starting point for further development of 
  writing skills.
\end{abstract}

\section{Introduction}

The aim of writing a paper is to infect the mind of your reader with
the brilliance of your idea~\cite{jones08}. 
The hope is that after reading your
paper, the audience will be convinced to try out your idea. In other
words, it is the medium to transport the idea from your head to your
reader's head. 
In the following
section, we show a common structure of scientific papers and briefly
outline some tips for writing good papers in
Section~\ref{sec:tips-writing}.

At that
point, it is important that the reader is able to reproduce your
work~\cite{schwab00,wavelab,gentleman05}. This is why it is also
important that if the work has a computational component, the software
associated with producing the results are also made available in a
useful form. Several guidelines for making your user's experience with
your software as painless as possible is given in
Section~\ref{sec:tips-software}.

This brief guide is by no means sufficient, on its own, to
make its reader an accomplished writer. The reader is urged to use the
references to further improve his or her writing skills.

\section{Models and Methods}
\label{sec:mod_meth}

There were two main parts of developing our ML system, data analysis and algorithmic design. While the first one focused on the nature, intricacies and interconnexions of the raw data and its preparation, the second one focused on the treatment of said data after refining. Let us now delve a bit further into those two aspects.

\subsection{Data Analysis and exploration}
With a dataset of 250'000 items and 30 features

\subsection{Algorithms used}


\section{Results}


\section*{Acknowledgements}
The author thanks Christian Sigg for his careful reading and helpful
suggestions.

\bibliographystyle{IEEEtran}
\bibliography{literature}

\end{document}
